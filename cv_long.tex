%%%%%%%%%%%%%%%%%%%%%%%%%%%%%%%%%%%%%%%%%
% Long Professional Curriculum Vitae
% LaTeX Template
% Version 1.1 (9/12/12)
%
% This template has been downloaded from:
% http://www.latextemplates.com
%
% Original author:
% Rensselaer Polytechnic Institute (http://www.rpi.edu/dept/arc/training/latex/resumes/)
%
% Important note:
% This template requires the res.cls file to be in the same directory as the
% .tex file. The res.cls file provides the resume style used for structuring the
% document.
%
%%%%%%%%%%%%%%%%%%%%%%%%%%%%%%%%%%%%%%%%%

%-------------------------------------------------------------------
%	PACKAGES AND OTHER DOCUMENT CONFIGURATIONS
%-------------------------------------------------------------------

\documentclass[11pt]{res} % Use the res.cls style, the font size can be changed to 11pt or 12pt here

\usepackage[
backend=biber,
style=authoryear,
sorting=ydnt,
]{biblatex}
\addbibresource{publications.bib}

\newcommand{\mysection}[1]{%
	\pagebreak[0]\section{#1}\vspace{1em}
}

\newenvironment{indentedblock*}[1][\parindent]
  {%
    \par
    \medskip
    \leftskip#1\relax
  }
  {%
    \par
    \medskip
  }

\newsectionwidth{0pt} % Stops section indenting

\begin{document}
\nocite{*}

%--------------------------------------------------------------------
%	NAME AND ADDRESS(ES) SECTION
%--------------------------------------------------------------------

\name{Charles ``Chuck" Wooters} % Your name at the top
\address{wooters@hey.com}

%--------------------------------------------------------------------
\begin{resume}

%--------------------------------------------------------------------
%	PROFESSIONAL EXPERIENCE SECTION
%--------------------------------------------------------------------
\vspace{2em}
\hrule

\mysection{EXPERIENCE}

%-------------------------------
\employer{\textbf{Dabkick, Inc}}
\title{{\sl VP AI, Products, and Engineering}}
\dates{Jan 2021 - Present}
\location{Cupertino, CA}
\begin{position}

\vspace{-.4in}

\begin{indentedblock*}[.1in]
{\sl Dabkick, Inc was acquired in July 2021. Details are currently confidential.}

At Dabkick, I have been leading the design and development of a conversational AI
for the connected TV domain. We are creating a speech-based entertainment
"concierge" that can: control a streaming television device, provide information
about movies and TV shows, and provide personalized recommendations.
\end{indentedblock*}

\end{position}



%------------------------------
\employer{\textbf{Microsoft/Semantic Machines}}
\title{{\sl Principal Researcher}}
\dates{2016-2021}
\location{Berkeley, CA}
\begin{position}

\vspace{-.4in}

\begin{indentedblock*}[.1in]

{\sl Semantic Machines was acquired by Microsoft in 2018.}

My main research focus was modeling human conversational behavior. As part of this effort, I built a prototype conversational AI framework that allowed real-time, full-duplex, human-computer dialogue (similar to Google's "Duplex" system). This framework served as a test-bed for the evaluation of research ideas in a realistic conversational setting. In addition to creating the (python-based) framework, I created prototypes  for several sub-components including: turn-taking, speaker identification, addressee detection. Other projects I worked on included: Dialogue-aware text-to-speech, GPT-2 based data-augmentation for dialogue systems

\end{indentedblock*}

\end{position}


%------------------------------
\employer{\textbf{Raytheon (Applied Signal Technology)}}
\title{{\sl Engineering Fellow}}
\dates{2015-2016}
\location{Sunnyvale, CA}
\begin{position}

\vspace{-.4in}
\begin{indentedblock*}[.1in]

I performed classified research in the area of human language technologies.

\end{indentedblock*}

\end{position}


%------------------------------
\pagebreak[3]

\employer{\textbf{Apple (Siri)}}
\title{{\sl Senior Software Engineer}}
\dates{2013-2015}
\location{Cupertino, CA}
\begin{position}

\vspace{-.4in}
\begin{indentedblock*}[.1in]

During my time at Apple, I made several contributions to Siri: I implemented a false-alarm mitigation algorithm for the “Hey Siri” wake-word, I created metrics for joint ASR/NL evaluation, I created NL annotation guidelines for semantic role labeling and intent classification, I refactored an incredibly convoluted tangle of bash/perl/python scripts that was responsible for running the nightly Siri LM updates.

\end{indentedblock*}

\end{position}


%------------------------------
\pagebreak[3]

\employer{\textbf{International Computer Science Inst.}}
\title{{\sl Senior Research Engineer}}
\dates{2000-2007, 2012-2013}
\location{Berkeley, CA}
\begin{position}

\vspace{-.4in}
\begin{indentedblock*}[.1in]

Some of the projects I worked on at ICSI included: Led ICSI's initial efforts in speaker diarization including participating in multiple NIST/DARPA evaluations, was a key contributor to ICSI and SRI's hybrid MLP/HMM ASR system, I designed and built an ad-hoc multi-microphone array for meeting rooms, I designed the initial database and annotation tools for the FrameNet lexical database project led by Berkeley Linguistics Prof. Charles Fillmore, and I ported the FrameNet database to the open-source NLTK project (https://www.nltk.org).

\end{indentedblock*}

\end{position}


%------------------------------
\pagebreak[3]

\employer{\textbf{Johns Hopkins University (HLT/COE)}}
\title{{\sl Adjunct Researcher}}
\dates{Summer 2009 and 2013}
\location{Baltimore, MD}
\begin{position}

\vspace{-.4in}
\begin{indentedblock*}[.1in]

I was a participant in the SCALE (Summer Camp for Applied Language Exploration) workshops. In 2009 I was focused on building ASR systems from large quantities of unlabelled data. In 2013 the focus was on domain-mismatch within an I-vector/PLDA speaker ID system.

\end{indentedblock*}

\end{position}

%------------------------------
\pagebreak[3]
\employer{\textbf{Next IT Corp.\ (now Verint)}}
\title{{\sl Chief Speech Scientist}}
\dates{2007-2012}
\location{Spokane, WA}
\begin{position}

\vspace{-.4in}

\begin{indentedblock*}[.1in]
At Next IT I was building automated virtual assistants for customer service applications. I led the transformation of the company's rules-based chatbot technology to a statistical machine learning-based approach. Additionally, I was responsible for the development of Next IT's IP strategy and portfolio. My technical contributions focused on the following areas: question answering, automatic intent classification, agglomerative clustering of user queries, and unsupervised topic modeling.

\end{indentedblock*}
\end{position}

%------------------------------
\pagebreak[3]

\employer{\textbf{BBN Technologies}}
\title{{\sl Senior Scientist}}
\dates{1999-2000}
\location{Boston, MA and Columbia, MD}
\begin{position}

\vspace{-.4in}
\begin{indentedblock*}[.1in]

I served as the liaison between the language technologies group at BBN and the U.S. Dept. of Defense. I also designed and tested various algorithmic modifications for a Mandarin ASR system as part of the NIST/DARPA evaluations.

\end{indentedblock*}

\end{position}


%------------------------------
\pagebreak[3]

\employer{\textbf{US Department of Defense}}
\title{{\sl Senior Computer Scientist}}
\dates{1993-1995, 1997-1999, 2009-2010}
\location{Fort Meade, MD}
\begin{position}

\vspace{-.4in}
\begin{indentedblock*}[.1in]

I performed classified research in the area of human language technologies.

\end{indentedblock*}

\end{position}



%------------------------------
\pagebreak[0]

\employer{\textbf{Computer Motion, Inc}}
\title{{\sl Sr.\ Member of Tech Staff}}
\dates{1995-1997}
\location{Goleta, CA}
\begin{position}

\vspace{-.4in}
\begin{indentedblock*}[.1in]

I developed an embedded speech recognition system for a surgical robotic arm based on the HTK ASR toolkit. The ASR system was an always-listening, small-vocabulary, speaker-dependent, command-and-control system for operating room environments. It was the first automatic speech recognition system to be granted US FDA approval for use on humans.

\end{indentedblock*}

\end{position}


%--------------------------------------------------------------------------
%	EDUCATION SECTION
%--------------------------------------------------------------------------
\pagebreak[2]
\mysection{EDUCATION}

\begin{tabular}{llcc}
  Ph.D.\ & Speech Recognition & UC Berkeley & 1993 \\[.5em]
  M.A.\ & Linguistics & UC Berkeley & 1988 \\[.5em]
  B.A.\ & Linguistics & UC Berkeley & 1986
\end{tabular}

 
%--------------------------------------------------------------------------
% SKILLS SECTION
%--------------------------------------------------------------------------
\mysection{SKILLS}

\begin{tabular}{ll}
Languages & Python, C, (some: Golang, Scala, Java, C++)  \\[.5em]
Revision Control & Git (GitHub, CodeCommit, GitLab) \\[.5em]
ML Tools & NumPy, SciPy, scikit-learn, pandas, PyTorch, TensorFlow, Kaldi \\[.5em]
Operating Systems &	Gnu/Linux (Ubuntu), Mac OS \\[.5em]
Cloud Services & AWS (cdk, lambda, s3, ec2, etc.), Azure (various cog services)

\end{tabular}
	

%--------------------------------------------------------------------------
%	ACTIVITIES SECTION
%--------------------------------------------------------------------------
\mysection{ACTIVITIES}

\begin{tabular}{ll}
  Reviewer & ICML, ICASSP, ICLSP, ASRU, SLT, Computer Speech \& Language \\[.5em]
  Demo Co-chair & 2016 Spoken Language Technology workshop \\[.5em]
  SLT Committee & Served on the Speech and Language Technical Committee from 2008-2012 \\[.5em]
  External Reviewer & Lawrence Livermore National Labs - Lab-directed R\&D 
\end{tabular}


%--------------------------------------------------------------------
%	PUBLICATIONS SECTION
%--------------------------------------------------------------------
% we need heading=bibnumbered here to tell biblatex to use \section 
% not \section* (which will produce a spurious * with this class)
%
\pagebreak[3]

\printbibliography[type=patent,title={PATENTS},heading=subbibliography]

\pagebreak[3]
\printbibliography[nottype=patent,title={PUBLICATIONS},heading=subbibliography]


%--------------------------------------------------------------------------

\end{resume}

\end{document}